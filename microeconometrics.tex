
%\documentclass{article}
\documentclass[16pt]{article}
\usepackage[tmargin=1in, bmargin=1in]{geometry}
\usepackage{amsmath, amsfonts, amsthm, amssymb, framed, enumerate}
\RequirePackage{parskip}

\setlength{\parindent}{0cm}

\newcommand{\bh}{\hat{\beta}}
\newcommand{\argmin}{\text{argmin}}
\newcommand{\norm}{\mathcal{N}}
\newcommand{\plim}{\text{plim}\; }
\newcommand{\Var}{\text{Var}}

\begin{document}

\title{Microecononometrics Handbook}
\author{J.T. Cho [joncho@seas.upenn.edu]}
\date{}

\maketitle

\begin{center}
A lightweight compilation of the important aspects of Petra Todd's advanced microeconometrics course at UPenn. I've taken the liberty to further explicate various derivations and proofs.
\end{center}

\section*{The Basics}

\textbf{PDF of Normal Distribution}
$$X \sim \mathcal{N}(\mu, \sigma^2)\;:\; f(X) = \frac{1}{\sigma \sqrt{2\pi}} \exp\bigg(-\frac{(X - \mu)^2}{2\sigma^2}\bigg)$$

\textbf{Variance of a Random Variable}
$$\text{Var}(X) = E[(X - \mu)^2] = E[X^2] - E^2[X]$$

\textbf{Independence of R.V.'s}
\begin{align*}
f(X, Y) =& f(X)f(Y) \\&\implies E[X|Y] = E[X]\\&\implies \text{Cov}(X,Y) = 0
\end{align*}

\textbf{Correlated R.V.'s}
$$\text{Cov}(X,Y) = E[XY] - \mu_X \mu_Y$$
If $\text{Cov}(X,Y) = 0$, $X, Y$ are uncorrelated.\\

\textbf{Properties of Estimators}
\begin{enumerate}[(i)]
  \item unbiasedness, $E[\hat{\mu_y}] = \mu_y$
  \item consistency, $\forall \epsilon > 0,\; \Pr(\mid \hat{\mu_y} - \mu_y \mid > \epsilon) \rightarrow 0 \text{ as } n\rightarrow\infty$
\end{enumerate}

\section*{Ordinary Least Squares (Simple)}

\textbf{Identifying Assumptions}

\begin{itemize}
\item nonstochastic $x_i$, s.t. only source of random variation in $y_i$ is $\epsilon_i$
\item mean-zero; $E[\epsilon] = \mathbf{0}$
\begin{itemize}
  \item $\epsilon \sim \mathcal{N}(0, \sigma^2 I)$
\end{itemize}
\item homoskedasticity; $E[\epsilon_i^2] = \sigma^2$
\item $\text{Cov}(\epsilon_i, \epsilon_j) = 0\;\forall\;i \neq j$
\end{itemize}

\textbf{Derivation of Matrix-formulation of OLS (Stochastic Regressors)}

$$\bh = \argmin_{\beta} \epsilon'\epsilon = \argmin_{\beta} (Y - X\beta)'(Y - X\beta)$$
\begin{align*}
  \frac{\partial}{\partial \beta} &= 2X' (Y - X \beta) = 0\\
    &X'Y - X'X\beta = 0\\
    &X'Y = X'X\beta\\
    &\beta = (X'X)^{-1} X'Y
\end{align*}

Since $\epsilon = Y - X\beta$, the F.O.C. implies that $X'\epsilon = 0$. If $X$ has a column of $1$'s corresponding to a constant term, then $X'$ in the expression implies that 
$$\sum_{i=1}^n \epsilon_i = 0$$
Then, the mean of the residuals is $0$. 

\textbf{Consistency of OLS Estimator}

We give the asymptotic distribution of $\bh - \beta$.

\begin{align*}
  \bh - \beta &= (X'X)^{-1}X'Y - \beta\\
              &= (X'X)^{-1} X'(X\beta + \epsilon) - \beta\\
              &= \beta + (X'X)^{-1}X'\epsilon - \beta\\
              &= (X'X)^{-1}X'\epsilon
\end{align*}
Since $\epsilon \sim \norm(0, \sigma^2 I)$, observe,
\begin{align*}
  \bh - \beta &\sim \norm(0, (X'X)^{-1} X' \sigma^2 I X(X'X)^{-1})\\
              &\sim \norm(0, \sigma^2 (X'X)^{-1}X'X(X'X)^{-1})\\
              &\sim \norm(0, \sigma^2(X'X)^{-1})
\end{align*}

\underline{Estimator for $\sigma^2$ for the variance-covariance matrix}

$$\hat{\sigma}^2 = \frac{1}{n-k} \sum_{i=1}^n \hat{\epsilon}_i^2$$

\textbf{Heteroskedasticity}

Relax assumption that $E[\epsilon_i^2] = \sigma^2$ for all $i$. Then, let $E[\epsilon \epsilon'] = \Omega$,
$$\Omega = \begin{pmatrix} 
  \sigma_1^2 & \dots & 0\\
  \vdots & \ddots & \vdots\\
  0 & \dots & \sigma_n^2
\end{pmatrix}$$

The robust var-cov estimator is $$\hat{\Sigma}_{robust} = (\sum_{i=1}^n x_i x_i')^{-1} \sum_{i=1}^n x_i x_i' \hat{\epsilon}_i^2(\sum_{i=1}^n x_i X_i')^{-1}$$

\section*{Generally Useful Techniques}

\textbf{Convergence}

Suppose we have the sequence of random variables $\{ X_i \}_{i=1}^n$ with sample mean $\frac{1}{n} \sum_{i=1}^n X_i$.

\underline{Convergence in probability}

For all $\epsilon > 0$, $$\lim_{n \to \infty} \Pr(\mid X_n - X \mid < \epsilon) = 1$$

Alternatively, is written $\plim X_n = X$.

Implies convergence in distribution.

\underline{Convergence in mean-square}

$$\lim_{n \to \infty} E[(X_n - X)^2] = 0$$

Implies convergence in probability.

\underline{Convergence almost surely}

For all $\epsilon > 0$, $$\lim_{n\to\infty} \Pr(\mid X_m - X\mid < \epsilon\;\forall\; m \geq n) = 1$$

In other words, $X_m$ converges to $X$ for some $n$ and stays small for all larger $n$.

Implies convergence in probability.

\underline{Convergence in distribution}

$$\lim_{n\to\infty} F_n(X_n) = F(X) \text{ for all $X$ such that $F(X)$ is continuous}$$ 

Let $F_i(x)$ be the CDF of $X$.

\textbf{Slutsky's Theorem}

If $X_n \to_D X$ and $A_n \to_P a$ for constant $a$, then

\begin{enumerate}[(i)]
  \item $A_n X_n \to_D aX$
  \item $A_n + X_n \to_D a + X$
  \item $\frac{X_n}{A_n} \to_D \frac{X}{a}$
\end{enumerate}

\textbf{Mann-Wald/Continuous-Mapping Theorem}

Let $g$ be a continuous function. Then,

\begin{enumerate}[(a)]
  \item $X_n \to_D X \implies g(X_n) \to_D g(X)$
  \item $X_n \to_P X \implies g(X_n) \to_P g(X)$
  \item $X_n \to_{A.S.} X \implies g(X_n) \to_{A.S.} g(X)$
\end{enumerate}

In other words, continuous functions are \emph{limit preserving} even if arguments are sequences of R.V.'s.

\textbf{Chebyshev's Weak Law of Large Numbers}

If $\{Y_i\}_i^n$ are i.i.d. with $E[Y_i] = \mu_Y, \Var(Y_i) = \sigma_Y^2 < \infty$, then $\bar{Y} \to_P \mu_Y$.

\textbf{Lindeberg-Levy Central Limit Theorem}
For i.i.d. R.V.'s $\{X_i\}_i^n$ with mean $\mu$ and variance $\sigma^2$, consider the sample mean
$$\bar{X}_n = \frac{1}{n} \sum_{i=1}^n X_i$$

$$E[\bar{X}_n] = \mu$$

$$\Var(\bar{X}_n) = \frac{\sigma^2}{n}$$

Then, defining the $Z$ statistic, $Z_n$,
$$Z_n = \frac{\bar{X}_n - \mu}{\sqrt{\frac{\sigma^2}{n}}}$$

The LLCLT states that the R.V.'s $Z_n \to_D \norm(0,1)$

\textbf{Law of Iterated Expectations}

$$E[X' \epsilon] = E[E[X' \epsilon]] = E[X' E[\epsilon \mid X]]$$

\section*{OLS with Stochastic Regressors}

\textbf{Identifying Assumptions}

Allow $X$ to be random. $\epsilon_i$ are i.i.d., with
\begin{align*}
  E[\epsilon_i\mid x_i] = 0, E[\epsilon_i^2\mid x_i] = \sigma^2\\
  E[x_i x_i'] < \infty, E[x_i x_i']^{-1} \text{ exists}\\
  E[\epsilon_i \epsilon_j \mid X] = 0\;\forall\; i \neq j
\end{align*}

\textbf{Unbiasedness}

Observe that expectations are taken over $x_i$ and $\epsilon_i$.
\begin{align*}
  E[\bh] &= E[(X'X)^{-1}X'Y]\\
    &= E[(X'X)^{-1}X'(X\beta + \epsilon)]\\
    &= E[\beta + (X'X)^{-1}X'\epsilon]\\
    &= \beta + E[(X'X)^{-1} X'\epsilon]\\
    &= \beta + E[(X'X)^{-1}X'E[\epsilon\mid X]]\\
    &= \beta
\end{align*}

\textbf{Asymptotic Distribution}

Consider the R.V. sequence $\{x_i \epsilon_i\}_{i=1}^n$.

Observe that $E[x_i \epsilon_i] = 0$, applying the L.I.E.
\begin{align*}
  \Var(x_i \epsilon_i) &= E[x_i^2 \epsilon_i^2]\\
      &= E[x_i^2 E[\epsilon_i^2 \mid x_i]]\\
      &= E[x_i x_i' \cdot \sigma^2]\\
      &= E[x_i x_i'] \sigma^2
\end{align*}

\begin{align*}
\bh - \beta &= (X'X)^{-1}(X'\epsilon)\\
  &= [\sum_{i=1}^n x_i x_i']^{-1}[\sum_{i=1}^n x_i\epsilon_i]\\
  &= [\frac{1}{n} \sum_{i=1}^n x_i x_i']^{-1}[\frac{1}{n}\sum_{i=1}^n x_i\epsilon_i]
\end{align*}
Observe that for any constant $c$, $(cX)^{-1} = \frac{1}{c}X^{-1}$.

Consider the following expression:
$$\sqrt{n}(\bh - \beta) = [\frac{1}{n} \sum_{i=1}^n x_i x_i']^{-1}[\frac{1}{n} \cdot \sqrt{n} \sum_{i=1}^n x_i\epsilon_i]$$
Applying the LLCLT to the second term gives 
$$\frac{1}{\sqrt{n}} \sum_{i=1}^n x_i \epsilon_i \sim \norm(0, \sigma^2 E[x_i x_i'])$$

Since $x_i'$ is a $1 \times k$ vector, the quantity $E[x_i x_i']$ is a matrix. Noting that by assumption $\plim \frac{1}{n} \sum_{i=1}^n x_i x_i' = E[x_i x_i']$ and applying Slutsky's theorem, 
\begin{align*}
  \sqrt{n}(\bh - \beta) &\sim \norm(0, \sigma^2 E[x_i x_i']^{-1} E[x_i x_i'] E[x_i x_i']^{-1})\\
  &\sim \norm(0, \sigma^2 E[x_i x_i']^{-1})
\end{align*}

(It is helpful to remember that convergence in probability implies convergence in distribution, which is necessary for the LLCLT).

\textbf{Consistency}

From the above expression for $\bh - \beta$, it immediately follows from the WLLN that the second term $\frac{1}{n} \sum_{i=1}^n x_i \epsilon_i \to_P 0$. Then, applying Slutsky's theorem, the product converges in probability to $0$.

\end{document}
